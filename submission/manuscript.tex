\documentclass[11pt,]{article}
\usepackage{lmodern}
\usepackage{amssymb,amsmath}
\usepackage{ifxetex,ifluatex}
\usepackage{fixltx2e} % provides \textsubscript
\ifnum 0\ifxetex 1\fi\ifluatex 1\fi=0 % if pdftex
  \usepackage[T1]{fontenc}
  \usepackage[utf8]{inputenc}
\else % if luatex or xelatex
  \ifxetex
    \usepackage{mathspec}
  \else
    \usepackage{fontspec}
  \fi
  \defaultfontfeatures{Ligatures=TeX,Scale=MatchLowercase}
\fi
% use upquote if available, for straight quotes in verbatim environments
\IfFileExists{upquote.sty}{\usepackage{upquote}}{}
% use microtype if available
\IfFileExists{microtype.sty}{%
\usepackage{microtype}
\UseMicrotypeSet[protrusion]{basicmath} % disable protrusion for tt fonts
}{}
\usepackage[margin=1.0in]{geometry}
\usepackage{hyperref}
\hypersetup{unicode=true,
            pdfborder={0 0 0},
            breaklinks=true}
\urlstyle{same}  % don't use monospace font for urls
\usepackage{graphicx,grffile}
\makeatletter
\def\maxwidth{\ifdim\Gin@nat@width>\linewidth\linewidth\else\Gin@nat@width\fi}
\def\maxheight{\ifdim\Gin@nat@height>\textheight\textheight\else\Gin@nat@height\fi}
\makeatother
% Scale images if necessary, so that they will not overflow the page
% margins by default, and it is still possible to overwrite the defaults
% using explicit options in \includegraphics[width, height, ...]{}
\setkeys{Gin}{width=\maxwidth,height=\maxheight,keepaspectratio}
\IfFileExists{parskip.sty}{%
\usepackage{parskip}
}{% else
\setlength{\parindent}{0pt}
\setlength{\parskip}{6pt plus 2pt minus 1pt}
}
\setlength{\emergencystretch}{3em}  % prevent overfull lines
\providecommand{\tightlist}{%
  \setlength{\itemsep}{0pt}\setlength{\parskip}{0pt}}
\setcounter{secnumdepth}{0}
% Redefines (sub)paragraphs to behave more like sections
\ifx\paragraph\undefined\else
\let\oldparagraph\paragraph
\renewcommand{\paragraph}[1]{\oldparagraph{#1}\mbox{}}
\fi
\ifx\subparagraph\undefined\else
\let\oldsubparagraph\subparagraph
\renewcommand{\subparagraph}[1]{\oldsubparagraph{#1}\mbox{}}
\fi

%%% Use protect on footnotes to avoid problems with footnotes in titles
\let\rmarkdownfootnote\footnote%
\def\footnote{\protect\rmarkdownfootnote}

%%% Change title format to be more compact
\usepackage{titling}

% Create subtitle command for use in maketitle
\newcommand{\subtitle}[1]{
  \posttitle{
    \begin{center}\large#1\end{center}
    }
}

\setlength{\droptitle}{-2em}

  \title{}
    \pretitle{\vspace{\droptitle}}
  \posttitle{}
    \author{}
    \preauthor{}\postauthor{}
    \date{}
    \predate{}\postdate{}
  
\usepackage{helvet} % Helvetica font
\renewcommand*\familydefault{\sfdefault} % Use the sans serif version of the font
\usepackage[T1]{fontenc}

\usepackage[none]{hyphenat}

\usepackage{setspace}
\doublespacing
\setlength{\parskip}{1em}

\usepackage{lineno}

\usepackage{pdfpages}

\begin{document}

\vspace{35mm}

\section{\texorpdfstring{Proton pump inhibitor administration does not
promote \emph{Clostridium difficile} colonization in a murine
model}{Proton pump inhibitor administration does not promote Clostridium difficile colonization in a murine model}}\label{proton-pump-inhibitor-administration-does-not-promote-clostridium-difficile-colonization-in-a-murine-model}

\vspace{35mm}

Sarah Tomkovich\({^1}\), Nicholas A. Lesniak\({^1}\), Yuan Li\({^1}\),
Lucas Bishop\({^1}\), Madison J. Fitzgerald\({^1}\), Patrick D.
Schloss\textsuperscript{1\(\dagger\)}

\vspace{40mm}

\(\dagger\) To whom correspondence should be addressed:
\href{mailto:pschloss@umich.edu}{\nolinkurl{pschloss@umich.edu}}

\(1\) Department of Microbiology and Immunology, University of Michigan,
Ann Arbor, MI 48109

\newpage

\linenumbers

\subsection{Abstract}\label{abstract}

Proton pump inhibitor (PPI) use has been associated with microbiota
alterations and susceptibility to \emph{Clostridium difficile}
infections (CDIs) in humans. We assessed how PPI treatment alters the
fecal microbiota and whether PPIs promote CDIs using a CDI mouse model.
Mice were treated with the PPI omeprazole for 7 days prior to \emph{C.
difficile} challenge and were compared to mice that were treated with
the antibiotic clindamycin or both omeprazole and clindamycin. We found
that omeprazole was not sufficient to promote \emph{C. difficile}
colonization. When omeprazole treatment was combined with the
antibiotic, there was a mixed level of \emph{C. difficile} colonization,
where one cage of mice remained resistant and the other cage was
colonized. 16S rRNA sequencing analysis revealed that omeprazole had
minimal impact on the murine microbiota throughout the 16 days of PPI
exposure. \emph{C. difficile} colonization of clindamycin-treated mice
was associated with changes in \emph{Alistipes}, \emph{Barnesiella},
Porphyromonadaceae, and Ruminococcaceae. These results suggest PPI
treatment alone is not sufficient to disrupt microbiota resistance to C.
difficle infection in mice that are normally resistant in the absence of
antibiotic treatment.

\subsection{Importance}\label{importance}

Antibiotics are a major risk factor for \emph{Clostridium difficile}
infections (CDIs), but it is less clear what other factors contribute to
CDI incidence and recurrence. Interestingly, other medications besides
antibiotics have been shown to alter the microbiota. Proton pump
inhibitors (PPIs) are associated with alterations in the human
intestinal microbiota through observational and interventional stduies
and PPI use has also been associated with CDIs. We evaluated the ability
of the PPI omeprazole to alter the murine intestinal microbiota and
disrupt colonization resistance to \emph{C. difficile}. We found PPI
treatment had minimal impact on the murine intestinal microbiota and was
not enough to promote \emph{C. difficile} colonization after challenge.
Further studies are needed to determine whether other factors such as
the composition of the starting bacterial community, comorbidities, and
use of additional medications contribute to the association between PPIs
and CDIs.

\newpage

\subsection{Introduction}\label{introduction}

Proton pump inhibitors (PPIs) are one of the most widely used drugs and
are available over the counter (1). PPIs are approved to treat a variety
of conditions including gastroesophageal reflux disease, ulcers, or as
prophylaxis for nonsteroidal anti-inflammatory drug (NSAID)-associated
gastrointestinal bleeding (2). Unfortunately, there have been a number
of adverse events associated with their use including small intestinal
bacterial overgrowth, decreased calcium absorption, and increased risk
of enteric infections (2). There's also been a number of large cohort
studies as well as interventional clinical trials that have demonstrated
specific alterations in the intestinal microbiome were associated with
PPI use (3, 4). PPI-associated microbiota changes have been attributed
to the ability of PPIs to increase stomach acid pH which may promote the
survival of oral and pathogenic bacteria (3, 4). Human fecal microbiota
changes with PPI use include increases in Enterococcaceae,
Lactobacillaceae, Micrococcaceae, Staphylococcaceae and Streptococcaceae
with decreases in Ruminococcaceae (3). An \emph{in vitro screen},
demonstrated that 3 PPIs inhibit the growth of 7-8 strains of
Lachnospiraceae and Ruminococcaceae, which represented
\textasciitilde{}16\% of the 40 intestinal bacteria strains that were
screened (5).\\
Antibiotics have a large impact on the intestinal microbiome and are a
primary risk factor for developing \emph{Clostridium difficile}
infections (CDIs) (6). It is less clear whether other human medications
that impact the microbiota also influence \emph{C. difficile}
colonization resistance. Multiple epidemiological studies have suggested
an association between PPI use and incidence or recurrence of CDIs (2,
4, 7, 8). However, most of the studies suggesting a link between PPIs
and \emph{C. difficile} are retrospective and did not evaluate the
microbiome (2, 7, 8). Thus, it is unclear whether the gastrointestinal
microbiome changes associated with PPI use play a role in the
association between PPIs and CDI incidence or recurrence. Additionally,
epidemiological studies have a limited capacity to address potential
confounders and comorbidities in patients that were on PPIs and
developed CDIs or recurrent CDIs (7, 8).\\
Although there have been a few \emph{C. difficile} mouse model studies
that have demonstrated PPIs have some effect on CDIs with or without
additional antibiotic treatment, none of these studies looked at the
interplay with the intestinal microbiome (9--11). Here, we evaluated the
impact of a PPI on the murine microbiome and susceptibility to \emph{C.
difficile} colonization in relation to clindamycin, an antibiotic that
perturbs the microbiome enough to allow \emph{C. difficile} to colonize
but is mild enough that \emph{C. difficile} is cleared within 10 days
(12).

\subsection{Results}\label{results}

\textbf{Murine fecal microbiomes were minimally affected by PPI
treatment } To test if PPI treatment alters the microbiome and promotes
susceptibility to CDIs, we gavaged mice daily with a high dose of
omeprazole for 7 days before \emph{C. difficile} challenge (Figure 1A)
and compared the PPI-treated mice to a group that received the
antibiotic, clindamycin, and a group that received both PPI and
clindamycin treatment. A principle coordinates analysis (PCoA) of the
Bray-Curtis distances over the initial 7 days of treatment revealed the
bacterial communities of PPI-treated mice remained relatively unchanged
(Figure 1B). Enterococcaceae, Lactobacillaceae, Micrococcaceae,
Staphylococcaceae, Streptococcaceae, and Ruminococcaceae are all familes
that have previously been impacted by PPIs in human studies, but we see
no significant fluctuations in the stool of our PPI-treated mice
throughout the course of the 16-day experiment (Figure 1C). In contrast,
the microbiomes from the 2 groups of mice that received clindamycin
started to shift away from the untreated and PPI-treated mice the day
after treatment (Figure 1D).

\textbf{PPI-treatment did not promote susceptibility to \emph{C.
difficile infection} in mice} After 7 days of PPI treatment or 1 day
after clindamycin treatment, mice were challenged with \emph{C.
difficile} 630 spores. While all 4 of the clindamycin-treated mice were
colonized with \emph{C. difficile}, all of the PPI-treated mice were
resistant to \emph{C. difficile} colonization (Figure 2A).
Interestingly, only 1 cage of the PPI and clindamycin-treated mice were
colonized, while the other cage was not (Figure 2A). PCoAs of the
bacterial communities from the 3 groups of mice after \emph{C.
difficile} challenge, showed that the greatest shifts in bacterial
communities occurred in the clindamycin-treated mice (Figure 2B). Within
5 days, all of the mice had cleared \emph{C. difficile}, suggesting
there was no difference in rate of clearance for the mice that were
initially colonized with \emph{C. difficile} despite continuing to
receive PPI treatment throughout the course of the experiment. Our
results suggest that PPI treatment alone had minor effects on microbial
community resistance to \emph{C. difficile} colonization in mice.
Instead most of the differences between our 3 treatment groups appear to
be driven by clindamycin administration and included decreased
\emph{Alistipes}, \emph{Barnesiella}, Porhyromonadaceae,
Ruminococcaceae, taxa previously found to be altered in
clindamycin-treated mice that were challenged with \emph{C. difficile}
(6).

\subsection{Discussion and
Conclusions}\label{discussion-and-conclusions}

Our results demonstrate the PPI omeprazole had minimal impact on the
stool microbiomes of mice and did not promote susceptibility to \emph{C.
difficile} colonization. In contrast, clindamycin-treated mice were
initially colonized, but cleared the infection within 10 days. Combining
PPI and clindamycin treatment had a mixed effect, one cage was colonized
with \emph{C. difficile} while the other cage remained resistant.\\
Our findings that PPI treatment had minimal impact on the fecal
microbiome are comparable to another PPI mouse study that indicated PPIs
have more of an effect on the small intestinal microbiota compared to
the fecal microbiota (13). We did not find significant changes for the
taxa observed to be significantly impacted by PPI use in human studies,
but a closer examination suggests that the significance of these
associations in humans is driven by some individuals more than others
and the effect size can be relatively small with overall differences on
a PCoA plot difficult to distinguish (14). Also, most of the studies to
date compare different individuals (PPI users versus non-users) or the
same individuals before and after treatment (3, 4), unfortunately these
limited microbiota snapshots may be skewed by day-to-day microbiota
variations (16) that would be revealed with more frequent longitudinal
sampling.\\
Previous work that administered the PPI lansoprazole daily for 2 weeks
to mice and then challenged with \emph{C. difficile} demonstrated that
PPI treatment alone resulted in detectable \emph{C. difficile} in the
stool 1 week after challenge, but also showed there was detectable
\emph{C. difficile} in mice not treated with antibiotics (10, 11). The
presence of \emph{C. difficile} in mice that were not treated with
either antibiotics or PPIs could be attributed to the higher dose of
vegetative cells (10\textsuperscript{8} CFU) and the different \emph{C.
difficile} strain used to challenge the mice and may partly explain
their observation of \emph{C. difficile} in PPI-treated mice (10, 11).
We have previously shown mice from are colony that are not given
antibiotics are resistant to \emph{C. difficile} 630 when challenged
with 10\textsuperscript{3} spores (17). Another group also used a higher
dose of 10\textsuperscript{6} spores or 10\textsuperscript{7} vegetative
cells and a different \emph{C. difficile} strain to challenge
antibiotic-treated mice or mice treated with both the PPI esomeprazole
for 2 days and antibiotics and demonstrated the antibiotic/PPI-treated
mice developed more severe CDIs (9).\\
Our study extended previous work examining PPIs and \emph{C. difficile}
in mice to examine the contribution of the intestinal microbiota. We
found the PPI omeprazole had minimal impact on the murine intestinal
microbiota and in contrast to previous work, PPIs did not promote
\emph{C. difficile} colonization. Our 16S rRNA sequencing suggested that
\emph{Streptococcus} and \emph{Enterococcus} are rare taxa in our
C57BL/6 mouse colony. If these taxa are important contributors to the
associations between PPIs and CDIs in humans, this may be part of the
reason we did not see an effect in our CDI mouse model. There could also
be differences at the species level for taxa that are altered with PPIs
in humans compared to mice, which may also contribute to the
association. Other murine starting communities or the addition of human
PPI treatment-associated strains may be needed to determine whether PPIs
can alter taxa and subsequently promote CDIs.\\
Factors such as age, body mass index, comorbidities, and use of other
medications in human studies may also be contributing to the association
between PPIs and CDIs. NSAID treatment can increase CDI severity in mice
(18). Given that PPIs are sometimes used in combination with NSAIDs (2),
it may be worth examining both medications together using a mouse model.
This study only looked at the impact of PPIs with or without
antibiotics, but future work may want to incorporate age, other
comorbidities and bacterial strains that are less common in mice to
evaluate whether PPIs in conjunction with other factors can increase the
risk of CDIs.

\subsection{Materials and Methods}\label{materials-and-methods}

\textbf{Animals} All mouse experiments were performed with 7- to
12-week-old C57BL/6 male and female mice. 2-3 mice were housed per cage
and male and female mice were housed separately. All animal experiments
were approved by the University of Michigan Animal Care and Use
Committee (IACUC) under protocol number PRO00006983.

\textbf{Drug treatments} Omeprazole (Sigma Aldrich) was prepared in a
vehicle solution of 40\% polyethylene glycol 400 (Sigma-Aldrich) in
phosphate buffered saline. Omeprazole was prepared from 20 mg/mL frozen
aliquots and diluted to an 8mg/mL prior to gavage. All mice received
\textasciitilde{}40 mg/kg omeprazole or vehicle solution once per day
through the duration of the experiment with treatment starting 7 days
before \emph{C. difficile} challenge (Figure 1A). One day prior to
\emph{C. difficile} challenge, mice received an intraperitoneal
injection of 10 mg/kg clindamycin or sterile saline vehicle. All drugs
were filter sterilized through a 0.22 micron syringe filter before
administration to animals.

\textbf{C. difficile infection model} Mice were challenged with \emph{C.
difficile} 630 seven days after the start of omeprazole treatment and
one day after clindamycin treatment. Mice were challenged with
10\textsuperscript{3} spores in ultrapure distilled water. Samples were
collected for 16S rRNA sequencing or \emph{C. difficile} CFU
quantification throughout the duration of the experiments at the
indicated timepoints (Figure 1A). Samples for 16S rRNA sequencing were
flash frozen in liquid nitrogen and stored at -80°C until DNA
extraction, while samples for CFU quantification were transferred into
an anaerobic chamber and serially diluted in PBS. Diluted samples were
plated on TCCFA (taurocholate, cycloserine, cefoxitin, fructose agar)
plates and incubated at 37°C for 24 hours under anaerobic conditions to
quantify \emph{C. difficile} CFU.

\textbf{16S rRNA gene sequencing} DNA for 16S rRNA gene sequencing was
extracted from 10-50 mg fecal pellet from each mouse using the DNeasy
Powersoil HTP 96 Kit (Qiagen) and an EpMotion 5075 automated pipetting
system (Eppendorf). The V4 hypervariable region of the 16S rRNA gene was
amplified with Accuprime Pfx DNA polymerase (Thermo Fisher Scientific)
using previously described custom barcoded primers (19). The 16S rRNA
amplicon library was sequenced with the MiSeq (Illumina). Amplicons were
cleaned up and normalized with the SequalPrep Normalization Plate Kit
(ThermoFisher Scientific) and pooled amplicons were quantified with the
KAPA library quantification kit (KAPA Biosystems).

\textbf{16S rRNA gene sequence analysis}Mothur (v1.40.5) was used for
all sequence processing steps (20) using the previously published
protocol (19). R (v.3.5.1) was used to generate figures and perform
statistical analysis.

\textbf{Statistical Analysis}

\textbf{Code availability} The code for all sequence processing and
analysis step as well as an Rmarkdown version of this manuscript is
available at
\url{https://github.com/SchlossLab/Tomkovich_PPI_XXXX_2019}.

\textbf{Data availability} The 16S rRNA sequencing data have been
deposited in the NCBI Sequence Read Archive (Accession no.
\_\_\_\_\_\_\_\_\_)

\newpage

\subsection{Figures}\label{figures}

\textbf{Figure 1. PPI treatment had minimal impact on the murine fecal
microbiota} A. Mouse experiment timeline and logistics. B. Principal
Coordinates Analaysis (PCoA) of Bray-curtis distances from stool samples
for all treatment groups during the intial 7 days of treatment with the
PPI omeprazole or vehicle cluster together and do not chage much over
time. C. Relative abundances of families previously associated with PPI
use in humans do not change much over the 16-day course of treatment
with PPIs (7 days before \emph{C. difficile} challenge through 9 days
post \emph{C. difficile} challenge). D. PCoA of stool samples collected
before and 1 day after antibiotic treatment, demonstrating the fecal
microbiota of mice treated with clindamycin starts to separate from the
rest of the samples.

\textbf{Figure 2. PPI treatment alone does not promote CDIs in mice} A.
C. difficile CFUs/g stool measured each day post C. difficile challenge
for clindamycin, clindamycin/PPI, and PPI-treated mice. B. PCoA of stool
samples collected after antibiotic treatments. C. Genera significantly
associated with treatment groups. D. Families significantly associated
with treatment groups.

\newpage

\subsection*{References}\label{references}
\addcontentsline{toc}{subsection}{References}

\hypertarget{refs}{}
\hypertarget{ref-Singh2018}{}
1. \textbf{Singh A}, \textbf{Cresci GA}, \textbf{Kirby DF}. 2018. Proton
pump inhibitors: Risks and rewards and emerging consequences to the gut
microbiome. Nutrition in Clinical Practice \textbf{33}:614--624.
doi:\href{https://doi.org/10.1002/ncp.10181}{10.1002/ncp.10181}.

\hypertarget{ref-nehra2018proton}{}
2. \textbf{Nehra AK}, \textbf{Alexander JA}, \textbf{Loftus CG},
\textbf{Nehra V}. 2018. Proton pump inhibitors: Review of emerging
concerns, pp. 240--246. \emph{In} Mayo clinic proceedings. Elsevier.

\hypertarget{ref-Imhann2017}{}
3. \textbf{Imhann F}, \textbf{Vila AV}, \textbf{Bonder MJ},
\textbf{Manosalva AGL}, \textbf{Koonen DP}, \textbf{Fu J},
\textbf{Wijmenga C}, \textbf{Zhernakova A}, \textbf{Weersma RK}. 2017.
The influence of proton pump inhibitors and other commonly used
medication on the gut microbiota. Gut Microbes \textbf{8}:351--358.
doi:\href{https://doi.org/10.1080/19490976.2017.1284732}{10.1080/19490976.2017.1284732}.

\hypertarget{ref-Naito2018}{}
4. \textbf{Naito Y}, \textbf{Kashiwagi K}, \textbf{Takagi T},
\textbf{Andoh A}, \textbf{Inoue R}. 2018. Intestinal dysbiosis secondary
to proton-pump inhibitor use. Digestion \textbf{97}:195--204.
doi:\href{https://doi.org/10.1159/000481813}{10.1159/000481813}.

\hypertarget{ref-maier2018extensive}{}
5. \textbf{Maier L}, \textbf{Pruteanu M}, \textbf{Kuhn M},
\textbf{Zeller G}, \textbf{Telzerow A}, \textbf{Anderson EE},
\textbf{Brochado AR}, \textbf{Fernandez KC}, \textbf{Dose H},
\textbf{Mori H}, \textbf{others}. 2018. Extensive impact of
non-antibiotic drugs on human gut bacteria. Nature \textbf{555}:623.

\hypertarget{ref-Schubert2015}{}
6. \textbf{Schubert AM}, \textbf{Sinani H}, \textbf{Schloss PD}. 2015.
Antibiotic-induced alterations of the murine gut microbiota and
subsequent effects on colonization resistance against clostridium
difficile. mBio \textbf{6}.
doi:\href{https://doi.org/10.1128/mbio.00974-15}{10.1128/mbio.00974-15}.

\hypertarget{ref-tariq2017association}{}
7. \textbf{Tariq R}, \textbf{Singh S}, \textbf{Gupta A}, \textbf{Pardi
DS}, \textbf{Khanna S}. 2017. Association of gastric acid suppression
with recurrent clostridium difficile infection: A systematic review and
meta-analysis. JAMA internal medicine \textbf{177}:784--791.

\hypertarget{ref-Elias2019}{}
8. \textbf{Elias E}, \textbf{Targownik LE}. 2019. The clinician's guide
to proton pump inhibitor related adverse events. Drugs
\textbf{79}:715--731.
doi:\href{https://doi.org/10.1007/s40265-019-01110-3}{10.1007/s40265-019-01110-3}.

\hypertarget{ref-hung2015proton}{}
9. \textbf{Hung Y-P}, \textbf{Ko W-C}, \textbf{Chou P-H}, \textbf{Chen
Y-H}, \textbf{Lin H-J}, \textbf{Liu Y-H}, \textbf{Tsai H-W}, \textbf{Lee
J-C}, \textbf{Tsai P-J}. 2015. Proton-pump inhibitor exposure aggravates
clostridium difficile--associated colitis: Evidence from a mouse model.
The Journal of infectious diseases \textbf{212}:654--663.

\hypertarget{ref-Kaur2007}{}
10. \textbf{Kaur S}, \textbf{Vaishnavi C}, \textbf{Prasad KK},
\textbf{Ray P}, \textbf{Kochhar R}. 2007. Comparative role of antibiotic
and proton pump inhibitor in ExperimentalClostridium difficileInfection
in mice. Microbiology and Immunology \textbf{51}:1209--1214.
doi:\href{https://doi.org/10.1111/j.1348-0421.2007.tb04016.x}{10.1111/j.1348-0421.2007.tb04016.x}.

\hypertarget{ref-kaur2011effect}{}
11. \textbf{Kaur S}, \textbf{Vaishnavi C}, \textbf{Prasad KK},
\textbf{Ray P}, \textbf{Kochhar R}. 2011. Effect of lactobacillus
acidophilus \& epidermal growth factor on experimentally induced
clostridium difficile infection. The Indian journal of medical research
\textbf{133}:434.

\hypertarget{ref-Jenior2018}{}
12. \textbf{Jenior ML}, \textbf{Leslie JL}, \textbf{Young VB},
\textbf{Schloss PD}. 2018. Clostridium difficile alters the structure
and metabolism of distinct cecal microbiomes during initial infection to
promote sustained colonization. mSphere \textbf{3}.
doi:\href{https://doi.org/10.1128/msphere.00261-18}{10.1128/msphere.00261-18}.

\hypertarget{ref-Yasutomi2018}{}
13. \textbf{Yasutomi E}, \textbf{Hoshi N}, \textbf{Adachi S},
\textbf{Otsuka T}, \textbf{Kong L}, \textbf{Ku Y}, \textbf{Yamairi H},
\textbf{Inoue J}, \textbf{Ishida T}, \textbf{Watanabe D}, \textbf{Ooi
M}, \textbf{Yoshida M}, \textbf{Tsukimi T}, \textbf{Fukuda S},
\textbf{Azuma T}. 2018. Proton pump inhibitors increase the
susceptibility of mice to oral infection with enteropathogenic bacteria.
Digestive Diseases and Sciences \textbf{63}:881--889.
doi:\href{https://doi.org/10.1007/s10620-017-4905-3}{10.1007/s10620-017-4905-3}.

\hypertarget{ref-Freedberg2015}{}
14. \textbf{Freedberg DE}, \textbf{Toussaint NC}, \textbf{Chen SP},
\textbf{Ratner AJ}, \textbf{Whittier S}, \textbf{Wang TC}, \textbf{Wang
HH}, \textbf{Abrams JA}. 2015. Proton pump inhibitors alter specific
taxa in the human gastrointestinal microbiome: A crossover trial.
Gastroenterology \textbf{149}:883--885.e9.
doi:\href{https://doi.org/10.1053/j.gastro.2015.06.043}{10.1053/j.gastro.2015.06.043}.

\hypertarget{ref-Imhann2015}{}
15. \textbf{Imhann F}, \textbf{Bonder MJ}, \textbf{Vila AV}, \textbf{Fu
J}, \textbf{Mujagic Z}, \textbf{Vork L}, \textbf{Tigchelaar EF},
\textbf{Jankipersadsing SA}, \textbf{Cenit MC}, \textbf{Harmsen HJM},
\textbf{Dijkstra G}, \textbf{Franke L}, \textbf{Xavier RJ},
\textbf{Jonkers D}, \textbf{Wijmenga C}, \textbf{Weersma RK},
\textbf{Zhernakova A}. 2015. Proton pump inhibitors affect the gut
microbiome. Gut \textbf{65}:740--748.
doi:\href{https://doi.org/10.1136/gutjnl-2015-310376}{10.1136/gutjnl-2015-310376}.

\hypertarget{ref-LlornsRico2019}{}
16. \textbf{Lloréns-Rico V}, \textbf{Raes J}. 2019. Tracking humans and
microbes. Nature \textbf{569}:632--633.
doi:\href{https://doi.org/10.1038/d41586-019-01591-y}{10.1038/d41586-019-01591-y}.

\hypertarget{ref-Jenior2017}{}
17. \textbf{Jenior ML}, \textbf{Leslie JL}, \textbf{Young VB},
\textbf{Schloss PD}. 2017. Clostridium difficile colonizes alternative
nutrient niches during infection across distinct murine gut microbiomes.
mSystems \textbf{2}.
doi:\href{https://doi.org/10.1128/msystems.00063-17}{10.1128/msystems.00063-17}.

\hypertarget{ref-Maseda2019}{}
18. \textbf{Maseda D}, \textbf{Zackular JP}, \textbf{Trindade B},
\textbf{Kirk L}, \textbf{Roxas JL}, \textbf{Rogers LM},
\textbf{Washington MK}, \textbf{Du L}, \textbf{Koyama T},
\textbf{Viswanathan VK}, \textbf{Vedantam G}, \textbf{Schloss PD},
\textbf{Crofford LJ}, \textbf{Skaar EP}, \textbf{Aronoff DM}. 2019.
Nonsteroidal anti-inflammatory drugs alter the microbiota and exacerbate
clostridium difficile colitis while dysregulating the inflammatory
response. mBio \textbf{10}.
doi:\href{https://doi.org/10.1128/mbio.02282-18}{10.1128/mbio.02282-18}.

\hypertarget{ref-Kozich2013}{}
19. \textbf{Kozich JJ}, \textbf{Westcott SL}, \textbf{Baxter NT},
\textbf{Highlander SK}, \textbf{Schloss PD}. 2013. Development of a
dual-index sequencing strategy and curation pipeline for analyzing
amplicon sequence data on the MiSeq illumina sequencing platform.
Applied and Environmental Microbiology \textbf{79}:5112--5120.
doi:\href{https://doi.org/10.1128/aem.01043-13}{10.1128/aem.01043-13}.

\hypertarget{ref-Schloss2009}{}
20. \textbf{Schloss PD}, \textbf{Westcott SL}, \textbf{Ryabin T},
\textbf{Hall JR}, \textbf{Hartmann M}, \textbf{Hollister EB},
\textbf{Lesniewski RA}, \textbf{Oakley BB}, \textbf{Parks DH},
\textbf{Robinson CJ}, \textbf{Sahl JW}, \textbf{Stres B},
\textbf{Thallinger GG}, \textbf{Horn DJV}, \textbf{Weber CF}. 2009.
Introducing mothur: Open-source, platform-independent,
community-supported software for describing and comparing microbial
communities. Applied and Environmental Microbiology
\textbf{75}:7537--7541.
doi:\href{https://doi.org/10.1128/aem.01541-09}{10.1128/aem.01541-09}.


\end{document}
